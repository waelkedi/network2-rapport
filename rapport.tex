\documentclass{article}
\usepackage[utf8]{inputenc}
\usepackage[frenchb]{babel}
\usepackage[T1]{fontenc}

\author{Benjamain André\\Alexis Lecocp\\Arnaud Collin\\Waelkens Dimitri}

\begin{document}


\Rapport (3 pages max.) contenant les éléments suivants:

\section{Description de votre Bunsiness Unit}

\section{Justification des choix liés à la configuration intra--domaine / inter-domaine (adressage, interfaces..)}

\section{Log des commandes utilisées sur vos routeurs Cisco}

R2#show running-config
Building configuration...

Current configuration : 612 bytes

version 12.2
service timestamps debug uptime
service timestamps log uptime
no service password-encryption

hostname R2

memory-size iomem 30
ip subnet-zero

ip dhcp database VLAN2

ip dhcp pool VLAN2
   network 10.2.1.0 255.255.255.0
   dns-server 8.8.8.8
   default-router 10.2.1.1

interface Ethernet0/0
 ip address 10.2.0.2 255.255.255.252
 half-duplex

interface Ethernet0/1
 ip address 10.2.1.1 255.255.255.0
 half-duplex

ip classless
ip route 10.2.0.0 255.255.255.252 10.2.0.1
ip route 10.2.0.0 255.255.255.252 Ethernet0/0
ip http server

line con 0
line aux 0
line vty 0 4

end

\section{Problèmes rencontrés}

Nous avions eu un premier problème lors de la mise en place du service DHCP. Nous avions testé plusieurs fois les commandes proposer dans l'énoncer du devoir sans succès. Ensuite, nous avions regardé dans la documentation CISCO en ligne qui proposait d'autres solutions.
En testant ces différentes solutions, nous avions remarqué que le router nous proposer des commandes (commande ?) qui ne reconnaisé pas. Nous avions supposés que le problème provenait du routeur et nous l'avions donc réinitialisé.Après avoir réinitialisé le routeur, nous avions mis en place le service DHCP avec les commandes proposer dans l'énoncer du devoir avec succès lors de la première tentative.
\section{Erreurs connues dans votre configuration}

\end{document}
